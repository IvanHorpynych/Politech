\documentclass[a4paper, 10pt]{article}

\usepackage[utf8x]{inputenc}
\usepackage[english, russian, ukrainian]{babel}
\usepackage{cmap}

\usepackage{graphicx}
\usepackage{float}

\usepackage{geometry}
\geometry{top = 2cm}
\geometry{bottom = 2cm}
\geometry{left = 3cm}
\geometry{right = 1.5cm}

\begin{document}
\begin{titlepage}
\begin{center}
\large{
Міністерство освіти і науки, молоді та спорту України\\
Національний технічний університет України\\
``Київський політехнічний інститут''\\
Факультет прикладної математики\\
Кафедра спеціалізованих комп’ютерних систем\\
}

\vfill

\large{\bf{
Лабораторна робота №1\\
Дисципліна:\\
``Архітектура комп'ютерів''\\
Тема:\\
``Арифметико--логічні пристрої з розподіленою логікою''\\
}}

\vfill

\begin{table}[h]
\centering
\begin{tabular}{lp{4cm}l}
Виконав:&&Перевірив:\\
Студент групи КВ--92&&Жабін В. І.\\
Гуль О. В.&&\\
Залікова книжка № КВ--9203&&\\
\end{tabular}
\end{table}

\vfill

Київ \the\year
\end{center}
\end{titlepage}
\newpage

\section{Мета}
Одержати навички в проектуванні арифметико--логічних пристроїв з розподіленою логікою і автоматів управління з жорсткою логікою.

\section{Завдання}
\begin{enumerate}
    \item Варіанти завдання визначаються молодшими розрядами $a_{7},\cdots,a_{1}$ двійкового номера залікової книжки.
    \item Розробити структурну схему операційного пристрою та змістовний мікроалгоритм обробки  додатних чисел відповідно до завдання наведеного у табл. 2.7. Для побудови схеми використати комбінаційний суматор, регістр--лічильник циклів та асинхронні регістри, що мають входи управління зсувами і занесенням інформації. На схемі повинні бути зазначені розрядність регістрів та шин.
    \item Розробити функціональну схему операційного пристрою.
    \item Виконати логічне моделювання роботи операційного пристрою за допомогою цифрової діаграми  із зазначеними викладачем значеннями операндів.
    \item Здійснити синтез пристрою управління, тип управляючого автомату обрати із табл. 2.9. Пам’ять автомата реалізувати на тригерах, тип яких обрати з табл. 2.8. Ураховувати, що мікрооперації на регістрах виконуються за перепадом управляючих сигналів з 1 в 0.
    \item Побудувати часові діаграми роботи автомата для кожної комбінацій значень логічних умов.
\end{enumerate}
Варіант: $9203=10001111110011_2$.\\
$a_{7},\cdots,a_{1}=1110011.$\\
Спосіб множення: 3.\\
Функція: $D=A(B+1)+0.5C.$\\
Тип автомата: Мура.\\
Тип тригера: T.\\
Ураховувати, що мікрооперації на регістрах виконуються за перепадом управляючих сигналів з 1 в 0.

\section{Теоретичні відомості}
TODO: Write it.

\section{Виконання завдання}
Нехай $n=4$. Z=YX.\\
RG1~-- регiстр результату Z.\\
RG2(1..n)~-- регiстр операнду X (множник).\\
RG3~-- регiстр операнду Y (множене).\\
SM~-- суматор.\\
CT~-- лiчильник циклiв.\\
Пiсля виконання алгоритму молодші розряди добутку будуть знаходитися в регістрі RG1, а старші~-- в регістрі RG2.
\begin{figure}[h!]
\begin{center}
\includegraphics[scale=0.5]{od_mul3.png}
\caption{Схема операційного пристрою для множення 3 способом.}
\end{center}
\end{figure}

\begin{figure}[H]
\begin{center}
\includegraphics[scale=0.3]{sum.png}
\caption{Функціональний мікроалгоритм множення 3 способом.}
\end{center}
\end{figure}

\noindent
X=RG2=0101$_{2}$=5\\
Y=RG3=0110$_{2}$=6\\
Z=YX=RG1 = 11110$_{2}$=30\\
\begin{table}[h!]
\centering
\begin{tabular}{|c|c|c|c|c|}
\hline
 CT &* RG2& RG1& RG3& №MO\\
\hline
0100&00101&0000&0110& 1\\
    &01010&0000&    & 2\\
0011&     &    &    & 4\\
    &10100&0000&    & 2\\
    &     &0110&    & 3\\
0010&     &    &    & 4\\
    &01000&1100&    & 2\\
0001&     &    &    & 4\\
    &10001&1000&    & 2\\
    &     &1110&    & 3\\
0000&     &    &    & 4\\
\hline
\end{tabular}
\caption{Дiаграма станiв регiстрiв при виконаннi алгоритму.}
\end{table}

\begin{table}[h!]
\centering
\begin{tabular}{|c|c|c|c|}
\hline
Елемент & Призначення & Мікрооперація & Управляючий сигнал\\
\hline
RG1 & Результат & Скидання & $\overline{R}$ \\
    &           & Зсув вліво & SL (<) $1\to0$ \\
    &           & Запис & W $1\to0$ \\
\hline
RG2 & Операнд X (множник) & Запис & W $1\to0$ \\
    &                     & Зсув влiво & SL (<) $1\to0$ \\
\hline
RG3 & Операнд Y (множене) & Запис & W $1\to0$ \\
\hline
CT & Лiчильник & Запис & W $1\to0$ \\
   &           & Декремент & $-$1 $1\to0$ \\
\hline
\end{tabular}
\caption{Перелік управляючих сигналів елементів.}
\end{table}

\begin{figure}[H]
\begin{center}
\includegraphics[scale=0.5]{fs_mul3.png}
\caption{Функціональна схема операційного пристрою.}
\end{center}
\end{figure}

\section{Висновки}

\end{document}
